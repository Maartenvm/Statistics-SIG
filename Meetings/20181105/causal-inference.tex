\documentclass{beamer}
\usepackage{amsfonts, comment}
\usepackage{tikz}
\usepackage{arydshln}

\usetheme{Berlin}
\usecolortheme{beaver}

\renewcommand{\P}{\mathbb{P}}
\DeclareMathOperator{\Do}{do}
\DeclareMathOperator{\Par}{par}
\newcommand{\Rec}{\text{Rec}}
\newcommand{\MedX}{\text{Med $A$}}
\newcommand{\Sex}{\text{Sex}}
\title{Intro to Causal Inference}
\author{Mees de Vries (BrainCreators)}
\date{\today}
\begin{document}

\maketitle

\begin{frame}
    \frametitle{Overview}
    \begin{enumerate}
        \item Motivation: why and what?
        \item Causal models: the definitions and working with them.
        \item Simpson's paradox: causal inference to the rescue!
    \end{enumerate}
\end{frame}

\section{Motivation}
\subsection{Why?}
\begin{frame}
    \frametitle{Why Causal Reasoning?}
    Causality is not built into our normal statistical language: random
    variables $X, Y, Z$ can be dependent, but there is no mathematical way to
    talk about which causes which.\pause\bigskip

    \begin{center}
        ``Correlation is not causation!''
    \end{center}
\end{frame}

\subsection{Example}
\begin{frame}
    \frametitle{Why Causal Reasoning?}
    Causality comes into play almost any time we want to take action
    based on statistical data. Consider:
    \begin{itemize}
        \item $X$: the frequency of high-sodium meals in my diet;
        \item $Y$: my blood pressure;
        \item $Z$: my headache frequency.\pause
    \end{itemize}
    We might very well have that $(X,Y)$ and $(Z, Y)$ are identically
    distributed. But we do not treat hypertension with painkillers!
\end{frame}

\begin{frame}
    \frametitle{Why Causal Reasoning?}
    \begin{itemize}
        \item $X$: high-sodium meal count;
        \item $Y$: blood pressure;
        \item $Z$: headache count.\pause
    \end{itemize}
    If we want to model how a better diet would lower my blood pressure, we
    might want to estimate
    \[
        \P(Y = y \mid X = x).
    \]
    However, the quantity
    \[
        \P(Y = y \mid Z = z)
    \]
    gives us no such information. Why is this difference not in our
    mathematical language?
\end{frame}

\subsection{Do-calculus}
\begin{frame}
    \frametitle{Do-calculus}
    Desideratum: an ``operator for interventions''. For $X, Y$ random variables
    we want an expression
    \[
        \P(Y = y \mid \Do(X = x)).\pause
    \]
    For $X, Y, Z$ as before we should have
    \[
        \P(Y = y \mid \Do(X = x)) = \P(Y = y \mid X = x),
    \]
    yet
    \[
        \P(Y = y \mid \Do(Z = z)) = \P(Y = y).
    \]
\end{frame}

\section{Causal Models}
\subsection{Example and Definition}
\begin{frame}
    \frametitle{Example}
    Our causal models should be as simple as possible. We just want to encode
    causal relationships.\pause\bigskip
    
    For our example:
    \begin{center}
        \begin{tikzpicture}
            \node[shape=circle,draw=black] (X) at (0,0) {X};
            \node[shape=circle,draw=black] (Y) at (2,0) {Y};
            \node[shape=circle,draw=black] (Z) at (4,0) {Z};
            \draw[->](X) edge (Y);
            \draw[->](Y) edge (Z);
        \end{tikzpicture}
    \end{center}\pause
    This says: $X$ is a direct cause of $Y$, and $Y$ is a direct cause of
    $Z$. \pause \alert{Implicit assumption}: there are no relationships between the
    variables that are not in the model.
\end{frame}

\begin{frame}
    \frametitle{Definition}
    \begin{block}{Definition}
    A \emph{causal model} consists of:
    \begin{itemize}
        \item A DAG $G = (V, E)$,
        \item independent random variables $(U_v)_{v \in V}$,
        \item functions $(f_v)_{v \in V}$ of the appropriate type.
    \end{itemize}
    The \emph{value} $X_v$ in a causal model $(G, U, f)$ at a node $v$ is
    defined recursively as
    \[
        X_v = f_v(U_v, X_{\Par(v)}).
    \]
    \end{block}
\end{frame}

\begin{frame}
    \frametitle{Definition}
    \begin{block}{Definition}
        A \emph{causal model} consists of $G = (V, E)$ with $(U_v)_{v \in V}$,
        $(f_v)_{v \in V}$. The value in the causal model at $v$ is $X_v
        = f_v(U_v, X_{\Par(v)})$.
    \end{block}
    Causal model for hypertension:
    \begin{center}
        \begin{tikzpicture}[scale=0.8]
            \node[shape=circle,draw=black] (X) at (0,0) {$X$};
            \node[shape=circle,draw=black] (Y) at (2,0) {$Y$};
            \node[shape=circle,draw=black] (Z) at (4,0) {$Z$};
            \node[shape=circle,draw=black,dashed] (UX) at (-1,1) {\footnotesize $U_X$};
            \node[shape=circle,draw=black,dashed] (UY) at (1,1) {\footnotesize $U_Y$};
            \node[shape=circle,draw=black,dashed] (UZ) at (3,1) {\footnotesize $U_Z$};
            \draw[->](X) edge (Y);
            \draw[->](Y) edge (Z);

            \draw[->](UX) edge (X);
            \draw[->](UY) edge (Y);
            \draw[->](UZ) edge (Z);
        \end{tikzpicture}
    \end{center}
    From now on, we omit the exogenous variables (the $U_v$) from our diagrams.
\end{frame}

\begin{frame}
    \frametitle{Dependence and independence}
    \begin{center}
        \begin{tikzpicture}[scale=0.6]
            \node[shape=circle,draw=black] (X) at (0,1) {\footnotesize$X$};
            \node[shape=circle,draw=black] (V) at (0,-1) {\footnotesize$V$};
            \node[shape=circle,draw=black] (Y) at (2,0) {\footnotesize$Y$};
            \node[shape=circle,draw=black] (Z) at (4,1) {\footnotesize$Z$};
            \node[shape=circle,draw=black] (W) at (4,-1) {\footnotesize$W$};
            \draw[->](X) edge (Y);
            \draw[->](V) edge (Y);
            \draw[->](Y) edge (Z);
            \draw[->](Y) edge (W);
        \end{tikzpicture}
    \end{center}
    Graphical models already tell us a lot about dependence and
    independence.\pause
    \begin{itemize}
        \item $X \perp V$
        \item $X \not\perp W$\pause
        \item $Z \not\perp W$\pause
        \item $Z \perp W \mid Y$\pause
        \item $X \not \perp V \mid Y$
    \end{itemize}
\end{frame}

\begin{frame}
    \frametitle{Dependence and independence}
    \begin{itemize}
        \item A node $v$ on an \emph{undirected} path in a \emph{directed} graph is
            a \alert{collider} if both adjacent edges on the path go into $v$.
        \item Given $Z \subseteq V$, a node $v$ on a path is
            \alert{$Z$-blocking} if it
            either
            \begin{itemize}
                \item is not a collider, and $v \in Z$, or
                \item is a collider, and neither $v$ nor any of its descendents
                    are in $Z$.
            \end{itemize}
    \end{itemize}
    \begin{theorem}
        $X \perp Y \mid (Z_1, Z_2, \ldots Z_n)$ if each undirected path $p$
        from $X$ to $Y$ has a $Z$-blocking node on it.
    \end{theorem}
\end{frame}

\begin{frame}
    \frametitle{Computations in Graphical Models}
    \begin{center}
        \begin{tikzpicture}[scale=0.6]
            \node[shape=circle,draw=black] (X) at (0,1) {\footnotesize$X$};
            \node[shape=circle,draw=black] (V) at (0,-1) {\footnotesize$V$};
            \node[shape=circle,draw=black] (Y) at (2,0) {\footnotesize$Y$};
            \node[shape=circle,draw=black] (Z) at (4,1) {\footnotesize$Z$};
            \node[shape=circle,draw=black] (W) at (4,-1) {\footnotesize$W$};
            \draw[->](X) edge (Y);
            \draw[->](V) edge (Y);
            \draw[->](Y) edge (Z);
            \draw[->](Y) edge (W);
        \end{tikzpicture}
    \end{center}
    Graphical models allow us to simplify computations.\pause
    \begin{align*}
        \P(z, w, y, x, v) &= \P(z \mid w, y, x, v)\P(w \mid y, x, v) \P(y \mid
        x, v) \P(x \mid v) \P(x)\\
        \only<3->{&= \P(z \mid y) \P(w \mid y) \P(y \mid x, v) \P(x) \P(v)}
    \end{align*}
\end{frame}

\begin{frame}
    \frametitle{Caution: Intransitive}
    Note that models are not transitive! The following are \alert{different}:
    \begin{center}
        \begin{tikzpicture}
            \node[shape=circle,draw=black] (X) at (0,0) {\footnotesize$X$};
            \node[shape=circle,draw=black] (Y) at (2,0) {\footnotesize$Y$};
            \node[shape=circle,draw=black] (Z) at (4,0) {\footnotesize$Z$};
            \draw[->](X) edge (Y);
            \draw[->](Y) edge (Z);
        \end{tikzpicture}
    \end{center}
    \begin{center}
        \begin{tikzpicture}
            \node[shape=circle,draw=black] (X) at (0,0) {\footnotesize$X$};
            \node[shape=circle,draw=black] (Y) at (2,1) {\footnotesize$Y$};
            \node[shape=circle,draw=black] (Z) at (4,0) {\footnotesize$Z$};
            \draw[->](X) edge (Y);
            \draw[->](Y) edge (Z);
            \draw[->](X) edge (Z);
        \end{tikzpicture}
    \end{center}
\end{frame}

\subsection{Causality}
\begin{frame}
    \frametitle{Causal Models and Bayesian Networks}
    So far, I have basically described Bayesian networks. What is the
    difference?\pause\bigskip

    In Bayesian networks, the following two are essentially the same model:
    \begin{center}
        \begin{tikzpicture}
            \node[shape=circle,draw=black] (X) at (0,0) {\footnotesize$X$};
            \node[shape=circle,draw=black] (Y) at (2,0) {\footnotesize$Y$};
            \node[shape=circle,draw=black] (Z) at (4,0) {\footnotesize$Z$};
            \draw[->](X) edge (Y);
            \draw[->](Y) edge (Z);
        \end{tikzpicture}
    \end{center}
    \begin{center}
        \begin{tikzpicture}
            \node[shape=circle,draw=black] (X) at (0,0) {\footnotesize$X$};
            \node[shape=circle,draw=black] (Y) at (2,0) {\footnotesize$Y$};
            \node[shape=circle,draw=black] (Z) at (4,0) {\footnotesize$Z$};
            \draw[->](Z) edge (Y);
            \draw[->](Y) edge (X);
        \end{tikzpicture}
    \end{center}
    This is obviously not what we want for causal networks!
\end{frame}

\begin{frame}
    \frametitle{Interventions}
    Recall that causality came into play with \alert{interventions}. What is an
    intervention in our model?\pause
    \alt<2>{
        \begin{center}
            \begin{tikzpicture}
                \node[shape=circle,draw=black] (X) at (0,0) {\footnotesize$X$};
                \node[shape=circle,draw=black] (Y) at (2,0) {\footnotesize$Y$};
                \node[shape=circle,draw=black] (Z) at (4,0) {\footnotesize$Z$};
                \draw[->](X) edge (Y);
                \draw[->](Y) edge (Z);
            \end{tikzpicture}
        \end{center}
    }{
        \begin{center}
            \begin{tikzpicture}
                \node[shape=circle,draw=black] (X) at (0,0) {\footnotesize$X$};
                \node[shape=circle,draw=black] (Y) at (2,0) {\footnotesize$y$};
                \node[shape=circle,draw=black] (Z) at (4,0) {\footnotesize$Z$};
                \draw[->](Y) edge (Z);
                \draw[->, dashed](X) edge (Y);

                \draw[red, thick] (0.75, -0.25) -- (1.25, 0.25);
                \draw[red, thick] (0.75, 0.25) -- (1.25, -0.25);
            \end{tikzpicture}
        \end{center}
    }
    Equations:
    \alt<2-3>{
    \begin{align*}
        X &= f_X(U_X)\\
        Y &= \alt<2-3>{f_Y(U_Y, X)}{y}\\
        Z &= \alt<2-3>{f_Z(U_Z, Y)}{f_Z(U_Z, Y') = f_Z(U_Z, y)}
    \end{align*}
    }{
    \begin{align*}
        X' &= f_X(U_X)\\
        Y' &= \alt<2-3>{f_Y(U_Y, X)}{y}\\
        Z' &= \alt<2-3>{f_Z(U_Z, Y)}{f_Z(U_Z, Y') = f_Z(U_Z, y)}
    \end{align*}
    }
    \pause\pause
    We use the $'$ to distinguish the random variables from the original and
    modified networks.
\end{frame}

\begin{frame}
    \frametitle{That $\Do$ Operator}
    \begin{definition}
        Let $(G, U, f)$ be a causal model with nodes $X, Y$. Then
        \[
            \P(Y = y \mid \Do(X = x))
        \]
        is the probability that $Y' = y$ in the modified network with $X$ set to
        $x$.
    \end{definition}
    \pause
    Example: hypertension case.
\end{frame}

\section{Simpson's Paradox}
\subsection{The Paradox}
\begin{frame}
    \frametitle{Simpson's Paradox}
    There is a particular medication $A$ on the market for disease $X$.
Statisticians want to test the effect of the medication. They randomly sample
people afflicted by $X$, and find:
{\small
\begin{center}
    \begin{tabular}{l|rr:r}
        Treatment & Recovery (male) & Recovery (female) & Recovery (total)\\
        \hline
        Medication $A$ & 9/10 = 90\%& 20/35 = 57\%& 29/45 = 64\% \\
        None           & 35/40 = 88\%& 8/15 = 53\%& 43/55 = 78\%\\
        \hdashline
        Total & 44/50 = 88\%& 28/50 = 54\%& 72/100 = 72\%
    \end{tabular}
\end{center}
}
\pause
\alert{Question}: should medication $A$ be banned?
\pause
\begin{itemize}
    \item The medication has a bad effect on the whole population...\pause
    \item but the medication has a good effect on both men and women!
\end{itemize}
\end{frame}

\begin{frame}
    \frametitle{Simpson's Paradox (alternative)}
    The statisticians decide to investigate further. They find 100 women who
have the early stages of the disease for a double blind study. They
additionally measure increase or decrease of a relevant protein $P$ before and after
treatment. 
{\small
\begin{center}
    \begin{tabular}{l|rr:r}
        Treatment & Recovery (inc) & Recovery (dec) & Recovery (total)\\
        \hline
        Medication $A$ &  1/10 = 10\%& 28/40 = 70\%& 29/50 = 58\%\\
        Placebo        &  9/25 = 36\%& 18/25 = 72\%& 27/50 = 54\%\\
        \hdashline
        Total          & 10/35 = 29\%& 46/65 = 71\%& 56/100 = 56\%
    \end{tabular}
\end{center}
}
\pause
\alert{Question}: should medication $A$ be banned (for women)?
\pause
\begin{itemize}
    \item The medication has a good effect on the whole population of women...\pause
    \item but the medication has a bad effect on both people whose $P$-levels
increase and decrease!
\end{itemize}
\end{frame}

\subsection{Causal models}
\begin{frame}
    \frametitle{Causal Model for SP/SPa}
    The causal models are different for the two cases. Clearly, medication
    usage or recovery outcome cannot \emph{cause} a difference in sex, so:
    \begin{center}
        \begin{tikzpicture}
            \node[shape=rectangle,draw=black] (X) at (0,0) {Sex};
            \node[shape=rectangle,draw=black] (Y) at (2,1) {Med $A$};
            \node[shape=rectangle,draw=black] (Z) at (4,0) {Rec};
            \draw[->](X) edge (Y);
            \draw[->](Y) edge (Z);
            \draw[->](X) edge (Z);
        \end{tikzpicture}
    \end{center}
    However, in our randomized trial, $P$-level cannot \emph{cause}
    a difference in medication usage, so:
    \begin{center}
        \begin{tikzpicture}
            \node[shape=rectangle,draw=black] (X) at (0,0) {Med $A$};
            \node[shape=rectangle,draw=black] (Y) at (2,1) {$P$-level};
            \node[shape=rectangle,draw=black] (Z) at (4,0) {Rec};
            \draw[->](X) edge (Y);
            \draw[->](Y) edge (Z);
            \draw[->](X) edge (Z);
        \end{tikzpicture}
    \end{center}
\end{frame}

\begin{frame}
    \frametitle{Interventions for SP}
    What if we make the intervention ``take medication'' in the original
    paradox?
    \alt<1>{
    \begin{center}
        \begin{tikzpicture}
            \node[shape=rectangle,draw=black] (X) at (0,0) {Sex};
            \node[shape=rectangle,draw=black] (Y) at (2,1) {Med $A$};
            \node[shape=rectangle,draw=black] (Z) at (4,0) {Rec};
            \draw[->](X) edge (Y);
            \draw[->](Y) edge (Z);
            \draw[->](X) edge (Z);
        \end{tikzpicture}
    \end{center}
    }{
    \begin{center}
        \begin{tikzpicture}
            \node[shape=rectangle,draw=black] (X) at (0,0) {Sex};
            \node[shape=rectangle,draw=black] (Y) at (2,1) {Med $A$: $\top$};
            \node[shape=rectangle,draw=black] (Z) at (4,0) {Rec};
            \draw[->, dashed, gray](X) edge (Y);
            \draw[->](Y) edge (Z);
            \draw[->](X) edge (Z);
        \end{tikzpicture}
    \end{center}
    }
    \pause\pause

    So we get:
    \begin{align*}
        \P(\text{Rec} &= \top \mid \Do(A = \top)) = \P(\Rec' = \top)\\
        &= \sum_{s \in \{M, F\}} \P(\text{Rec}' = \top \mid \Sex' = s)\P(\Sex'
        = s)\\
        &= \sum_{s \in \{M, F\}}\P(\Rec = \top \mid A = \top, \Sex = s)\P(\Sex
        = s).
    \end{align*}
\end{frame}

\begin{frame}
    \frametitle{Interventions for SPa}
    What about the alternate paradox?
    \alt<1>{
    \begin{center}
        \begin{tikzpicture}
            \node[shape=rectangle,draw=black] (X) at (0,0) {Med $A$};
            \node[shape=rectangle,draw=black] (Y) at (2,1) {$P$-level};
            \node[shape=rectangle,draw=black] (Z) at (4,0) {Rec};
            \draw[->](X) edge (Y);
            \draw[->](Y) edge (Z);
            \draw[->](X) edge (Z);
        \end{tikzpicture}
    \end{center}
    }{
    \begin{center}
        \begin{tikzpicture}
            \node[shape=rectangle,draw=black] (X) at (0,0) {Med $A = \top$};
            \node[shape=rectangle,draw=black] (Y) at (2,1) {$P$-level};
            \node[shape=rectangle,draw=black] (Z) at (4,0) {Rec};
            \draw[->](X) edge (Y);
            \draw[->](Y) edge (Z);
            \draw[->](X) edge (Z);
        \end{tikzpicture}
    \end{center}
    }
    \pause\pause
    So we get:
    \begin{align*}
        \P(\Rec = \top \mid \Do(A = \top)) &= \P(\Rec' = \top)\\
        &= \P(\Rec = \top \mid A = \top)
    \end{align*}
    So do not separately take the $P$-level cases into account!
\end{frame}

\subsection{Conclusion}
\begin{frame}
    \frametitle{Take-aways}
    \begin{itemize}
        \item Basic causal modeling might not directly lead to new insight, but
            it helps us formalize our intuitions.
        \item Several counterintuitive problems become nicer in the context of
            causal modeling. 
            \begin{itemize}
                \item Simpson's paradox
                \item Berkson's paradox
                \item Monty Hall problem
            \end{itemize}
        \item Given a causal model, you can often answer questions about
            interventions without making interventions.
        \item Major drawback: we require a DAG. Many real-world complex systems
            have feedback, and are thus not acyclic.
    \end{itemize}
\end{frame}

\end{document}
